\documentclass[12pt]{article}
\usepackage[top=1in, bottom=1in, left=1in, right=1in]{geometry}
\usepackage{paralist}
\usepackage{graphicx}
\usepackage{titlesec}  
\usepackage{amssymb}
\usepackage{amsmath} 

\begin{document}
\title{STAT 498 B Marbles Simulation}
\author{Nan Tang 1662478}
\date{\today}
\maketitle

\section*{Q1}
\begin{verbatim}
#Define a marble bag of 40 red and 40 clear. 
#Simulate 1000 draws.
marbleBag <- rep(c("red", "clear"), 40)
draws <- 1000

#Sample size of 14 from a bag of 50% read and 50% clear marbles.
sampleSize <- 14
extremeEvents <- 0
for (i in 1 : draws) {
  marbleResult <- sample(marbleBag, sampleSize, replace = FALSE)
  if (all(marbleResult == "red") | all(marbleResult == "clear")) {
    extremeEvents <- extremeEvents + 1
  }
}
extremeEvents / draws
## Result is 0 ##
\end{verbatim}

\noindent The result of this simulation conveys that the probability of having exactly half red and half clear marbles in 1000 draws of 14 marbles each is 0. 


\section*{Q2}
\begin{verbatim}
#Simulate 1000 draws with sample size of 28 from same marble bag.
sampleSize <- 28
extremeEvents <- 0
for (i in 1 : draws) {
  marbleResult <- sample(marbleBag, sampleSize, replace = FALSE)
  if (all(marbleResult == "red") | all(marbleResult == "clear")) {
    extremeEvents <- extremeEvents + 1
  }
}
extremeEvents / draws
## Result is 0 ##
\end{verbatim}

\noindent This simulation demonstrates that the probability of having exactly half red half clear marbles in 1000 draws of 28 marbles each is also 0. 


\section*{Q3}
\begin{verbatim}
#Simulate 1000 draws with sample size of 4 from same marble bag, while 
#success is defined to be exactly half red and half clear. 
sampleSize <- 4
successEvents <- 0
for (i in 1 : draws) {
  marbleResult <- sample(marbleBag, sampleSize, replace = FALSE)
  if (sum(marbleResult == "red") == 2) {
    successEvents <- successEvents + 1
  }
}
successEvents / draws
## Result is 0.38 ##
\end{verbatim}

\noindent This simulation show that within 1000 draws of 4 marbles each, the probability of getting half red and half clear is 0.38.


\section*{Q4}
\begin{verbatim}
#This function calculates the proportion of schools which have over 3% of 
#students scores top 1%.
distributionTestFun <- function(simulations, groupSize) {
  successProb = 0.01
  extremeProb = 0.03
  groupRandom <- rbinom(simulations, p = successProb, size = groupSize)
  groupProb <- groupRandom / groupSize
  extremeCount <- sum(groupProb > extremeProb)
  return(extremeCount / simulations)
}

#Calculate proportion of schools which have over 3% students scores top 1% 
#within 1000 Group A schools.
group_a <- distributionTestFun(1000, 1000)
group_a
## Result is 0 ##

#Calculate proportion of schools which have over 3% students scores top 1% 
#within 1000 Group B schools.
group_b <- distributionTestFun(1000, 100)
group_b
## Result is 0.016 ##
\end{verbatim}

\noindent The result of the simulation demonstrates that in groups of different school size, the proportions of schools that have over $3 \%$ top students vary largely. In group A, which contains 1000 school of 1000 students each, $0 \%$ of these schools have over $3 \%$ top students. In group B, which is consisted of schools of size 100, shows $1.6 \%$ of these schools have over $3 \%$ top students. 


\section*{Q5}
\begin{verbatim}
#This function calculates the proportion of schools which have an average score below 155.
distributionTestFun2 <- function(simulations, groupSize) {
  scoreMean <- 160
  stanError <- 40 / sqrt(groupSize)
  groupRandom <- rnorm(simulations, mean = scoreMean, sd = stanError)
  extremeCount <- sum(groupRandom < 155)
  return(extremeCount / simulations)
}

group_a_extreProb <- distributionTestFun2(1000, 1000)
group_a_extreProb
## Result is 0 ##

group_b_extreProb <- distributionTestFun2(1000, 100)
group_b_extreProb
## Result is 0.104 ##
\end{verbatim}

\noindent Within these two groups of schools with normal distributed score, $0 \%$ schools of group A which is consisted of schools of 1000 students have an average score below 155, while $10.4 \%$ schools of group B which is consisted of schools of 100 students have an average score below 155. 

\end{document}